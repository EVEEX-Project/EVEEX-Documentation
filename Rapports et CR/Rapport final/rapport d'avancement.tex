% Options for packages loaded elsewhere
\PassOptionsToPackage{unicode}{hyperref}
\PassOptionsToPackage{hyphens}{url}
%
\documentclass[
  french,
]{article}
\usepackage{amsmath,amssymb}
\usepackage{lmodern}
\usepackage{ifxetex,ifluatex}
\ifnum 0\ifxetex 1\fi\ifluatex 1\fi=0 % if pdftex
  \usepackage[T1]{fontenc}
  \usepackage[utf8]{inputenc}
  \usepackage{textcomp} % provide euro and other symbols
\else % if luatex or xetex
  \usepackage{unicode-math}
  \defaultfontfeatures{Scale=MatchLowercase}
  \defaultfontfeatures[\rmfamily]{Ligatures=TeX,Scale=1}
\fi
% Use upquote if available, for straight quotes in verbatim environments
\IfFileExists{upquote.sty}{\usepackage{upquote}}{}
\IfFileExists{microtype.sty}{% use microtype if available
  \usepackage[]{microtype}
  \UseMicrotypeSet[protrusion]{basicmath} % disable protrusion for tt fonts
}{}
\makeatletter
\@ifundefined{KOMAClassName}{% if non-KOMA class
  \IfFileExists{parskip.sty}{%
    \usepackage{parskip}
  }{% else
    \setlength{\parindent}{0pt}
    \setlength{\parskip}{6pt plus 2pt minus 1pt}}
}{% if KOMA class
  \KOMAoptions{parskip=half}}
\makeatother
\usepackage{xcolor}
\IfFileExists{xurl.sty}{\usepackage{xurl}}{} % add URL line breaks if available
\IfFileExists{bookmark.sty}{\usepackage{bookmark}}{\usepackage{hyperref}}
\hypersetup{
  pdftitle={"Rapport final : EVEEX"},
  pdfauthor={; ; ; ; },
  pdflang={fr},
  pdfkeywords={encodage, python, C, FPGA, video, codec, x264},
  hidelinks,
  pdfcreator={LaTeX via pandoc}}
\urlstyle{same} % disable monospaced font for URLs
\setlength{\emergencystretch}{3em} % prevent overfull lines
\providecommand{\tightlist}{%
  \setlength{\itemsep}{0pt}\setlength{\parskip}{0pt}}
\setcounter{secnumdepth}{-\maxdimen} % remove section numbering
\ifxetex
  % Load polyglossia as late as possible: uses bidi with RTL langages (e.g. Hebrew, Arabic)
  \usepackage{polyglossia}
  \setmainlanguage[]{french}
\else
  \usepackage[main=french]{babel}
% get rid of language-specific shorthands (see #6817):
\let\LanguageShortHands\languageshorthands
\def\languageshorthands#1{}
\fi
\ifluatex
  \usepackage{selnolig}  % disable illegal ligatures
\fi

\title{"Rapport final : EVEEX"}
\usepackage{etoolbox}
\makeatletter
\providecommand{\subtitle}[1]{% add subtitle to \maketitle
  \apptocmd{\@title}{\par {\large #1 \par}}{}{}
}
\makeatother
\subtitle{"Encodage Vidéo ENSTA Bretagne Expérimental"}
\author{A. Froehlich \and G. Leinen \and J.N. Clink \and H. Saad \and H. Questroy}
\date{"05-01-2021"}

\begin{document}
\maketitle

\hypertarget{header-n2}{%
\section{Informations}\label{header-n2}}

Membres du groupe : Guillaume Leinen, Jean-Noël Clink, Hussein Saad,
Alexandre Froehlich, Hugo Questroy

Encadrants : Pascal Cotret, Jean-Christophe Le Lann, Joël Champeau

Code disponible sur GitHub ==\textgreater{}
\url{https://github.com/EVEEX-Project}

Eportfolio Mahara disponible =\textgreater{}
\url{https://mahara.ensta-bretagne.fr/view/groupviews.php?group=348}

\hypertarget{header-n7}{%
\section{Abstract}\label{header-n7}}

\hypertarget{header-n9}{%
\section{Résumé}\label{header-n9}}

\textbackslash pagebreak

\hypertarget{header-n11}{%
\section{Remerciements}\label{header-n11}}

\begin{itemize}
\item
  Nos encadrants \textbf{Pascal Cotret}, \textbf{Jean-Christophe Le
  Lann} et \textbf{Joël Champeau}, qui nous aident à définir les
  objectifs à atteindre, nous prêtent le matériel nécessaire en
  particulier les cartes FPGA, ainsi qu'a résoudre des problèmes
  théoriques.
\item
  \textbf{Enjoy Digital}, société créée par un Alumni Ensta-Bretagne, et
  son produit Litex qui nous sera très utile sur l'implémentation
  hardware.
\item
  Le site \textbf{FPGA4students} pour ses tutoriels VHDL/Verilog.
\item
  \textbf{Jean-Christophe Leinen} pour ses conseils sur les méthodes
  Agiles.
\end{itemize}

\textbackslash pagebreak

\hypertarget{header-n22}{%
\section{Introduction}\label{header-n22}}

Dans son rapport annuel sur l'usage d'internet, cisco met en exergue
l'importance du traffic vidéo dans l'internet mondial {[}1{]} : la part
du streaming dans le débit mondial ne fait qu'augmenter, les définitions
de lecture augmentent elles aussi, ainsi que la part de "live" très
gourmands en bande passante.

Dans cette perspective, il est clair que l'algorithme de compression
utilisée pour compresser un flux vidéo brut à toute son importance, le
moindre \% de bande passante économisée permettant de libérer plusieurs
TB/s de bande passante. Ces codecs sont relativmeent peu connu du grand
public, en voici quelques uns:

\begin{itemize}
\item
  MPEG-4 (H.264) : il s'agit d'un des codecs les plus connu, car il
  génère des fichiers d'extension .mp4 et est embarqué dans un grand
  nombre d'appareils. Il est important de savoir que, comme le H.265, ce
  codec est \textbf{protégé} par un brevet, et les services et
  constructueurs souhaitant utilisé cet algorithme ou un connecteur basé
  sur l'algorithme doivent reverser des royalties à MPEG-LA {[}2{]}
  (\emph{La fabrication d'un connecteur display-port coutant 0.20\$ de
  license au constructeur}), coalitions de plusieurs entreprises du
  numériques comme sony, panasonic ou encore l'université de Columbia.
\item
  VPx : appartenant à l'origine à One-technologie, l'entreprise fut
  racheté par google à la suite. Les codecs VP (dernière version VP9)
  sont ouvert et sans royalties. Ce codec est plutot performant en terme
  de compression mais son encodage est lent {[}3{]}, avec sur un core i7
  d'intel en 720p une vitesse de compression proche de 2 images/s,
  occasionnant des couts non négligable en puissance informatique pour
  les entreprises productrices de contenu (comme Netflix).
\item
  H.265 : l'un des codecs les plus récents, il permet une réduction
  significative de la bande passante necessaire, nottament pour le
  streaming, mais est aussi lent à l'encodage, et demande des royalties.
\end{itemize}

Vous l'aurez constaté, les codecs les plus actuels sont souvent détenus
par des entreprises du secteur. Pour limiter les coûts annexes pour les
entreprises, un consortium s'est créé en 2015, a but non lucratif, afin
de developper un codec libre de droit aussi efficace que les autres :
l'\textbf{Aliance for Open-Media} {[}4{]}. On compte la plupart des
acteurs du secteur dans ce consortium, nottament l'arrivée des acteurs
du streaming comme Hulu ou Netflix. Leur création, le codec AV1, basé
sur VP9, est donc libre de droit, et est nottament très employé dans le
streaming vidéo. Il a l'avantage de proposer une compression 30\% plus
forte que le H265 {[}5{]}, mais occasionne par les différentes
bibliothèques utilisées une utilisation des ressources infiormatiques
(puissance CPU) bien plus importante, aussi bien du coté encodeur que
décodeur. La transition vers l'AV1 sur les grandes pateformes videos
(Netflix, Youtube) n'est pas encore réalisé mais bien en cours de
planification.

\textbackslash pagebreak

\hypertarget{header-n34}{%
\section{\texorpdfstring{Présentation de la problématique
}{Présentation de la problématique }}\label{header-n34}}

\hypertarget{header-n37}{%
\subsection{\texorpdfstring{Réalisation
}{Réalisation }}\label{header-n37}}

\hypertarget{header-n38}{%
\subsection{\texorpdfstring{Difficultés rencontrées
}{Difficultés rencontrées }}\label{header-n38}}

\textbackslash pagebreak

\hypertarget{header-n41}{%
\section{Annexes}\label{header-n41}}

\textbackslash pagebreak

\hypertarget{header-n43}{%
\section{Bibliographie}\label{header-n43}}

\textbf{{[}1{]}}
https://www.cisco.com/c/en/us/solutions/collateral/executive-perspectives/annual-internet-report/white-paper-c11-741490.html

\textbf{{[}2{]}} mpeg-la royalties
https://www.businesswire.com/news/home/20150305006071/en/MPEG-LA-Introduces-License-DisplayPort

\textbf{{[}3{]}} encodage lent VP9
https://groups.google.com/a/webmproject.org/g/codec-devel/c/P5je-wvcs60?pli=1

\textbf{{[}4{]}} http://aomedia.org/about/

\textbf{{[}5{]}}
https://www.lesnumeriques.com/tv-televiseur/av1-nouveau-standard-video-adopte-a-unanimite-n73213.html

\textbf{{[}6{]}} 2014. \emph{Vivado Design Suite Tutorial : High-Level
Synthesis}. {[}ebook{]} Available at:
\href{https://www.xilinx.com/support/documentation/sw_manuals/xilinx2014_2/ug871-vivado-high-level-synthesis-tutorial.pdf}{https://www.xilinx.com/support/documentation/sw\emph{manuals/xilinx2014}2/ug871-vivado-high-level-synthesis-tutorial.pdf}
{[}Accessed 15 October 2020{]}.

\textbf{{[}7{]}} LiteX-hub - Collaborative FPGA projects around LiteX.
Available at: \url{https://github.com/litex-hub} {[}Accessed 1 December
2020{]}.

\textbf{{[}8{]}} Zenhub.com - project Management in Github. Available
at: \url{https://www.zenhub.com/} {[}Accessed 15 october 2020{]}

\textbf{{[}9{]}} Pilai, L., 2003. \emph{Huffman Coding}. {[}pdf{]}
Available at:
\url{https://www.xilinx.com/support/documentation/application_notes/xapp616.pdf}
{[}Accessed 1 December 2020{]}.

\textbackslash pagebreak

\hypertarget{header-n54}{%
\section{Glossaire}\label{header-n54}}

\textbf{Flux-Vidéo:} Processus d'envoi, réception et lecture en continu
de la vidéo.

\textbf{Open-Source:} Tout logiciel dont les codes sont ouverts
gratuitement pour l'utilisation ou la duplication, et qui permet de
favoriser le libre échange des savoirs informatiques.

\textbf{RGB:} Rouge, Vert, Bleu. C'est un système de codage informatique
des couleurs. Chaque pixel possède une valeur de rouge, une de vert et
une de bleu.

\textbf{FPGA:} Une puce FPGA est un circuit imprimé reconfigurable
fonctionnant à base de portes logiques.

\textbf{Macrobloc }(ou Macroblock) : Une partie de l'image de taille
16x16 pixels.

\textbf{Bitstream:} Flux de données en binaire.

\textbf{VHDL:} Langage de description de matériel destiné à représenter
le comportement ainsi que l'architecture d'un système électronique
numérique.

\textbf{IDE:} Environnement de Développement.

\textbf{Frames:} Images qui composent une vidéo. On parle de FPS (Frames
Per Second) pour mesurer la fréquence d'affichage.

\textbf{HLS}: High Level Synthesis. Outil logiciel permettant de
synthétiser du code haut niveau en un code de plus bas niveau.

\textbf{SoC}: System on Chip. Puce de sillicium intégrant plusieurs
composants, comme de la mémoire, un processeur, et un composant de
gestion d'entrées/sorties.

\end{document}
